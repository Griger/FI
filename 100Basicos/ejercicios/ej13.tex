\subsection{Ejercicio 13}
\def\parte{enunciados}
\ifx\capitulo\parte
Para ciertos ángulos el seno y el coseno coinciden. Esto se produce cuando el seno y el coseno valen $\pm \frac{\sqrt{2}}{2}$. Para comprobarlo dibuja las funciones seno y coseno en el intervalo $[0, 10\pi]$, y añade también dos líneas horizontales a las alturas $\frac{\sqrt{2}}{2}$ y $-\frac{\sqrt{2}}{2}$, dándole distinto color y estilo de línea a cada una. Además, añade una leyenda que permita identificar el seno y el coseno en la gráfica.
\fi

\def\parte{recetas}
\ifx\capitulo\parte
\begin{enumerate}
\item Declaramos un rango de valores para la $x$ que vaya de 0 a $10\pi$, con saltos lo suficientemente pequeños como para que las gráficas se vean con suficiente resolución.
\item Dibujamos cada una de las grácias.
\item Las líneas rectas son funciones que para cualquier $x$ devuelve el valor de su altura ($\frac{\sqrt{2}}{2}$ y $-\frac{\sqrt{2}}{2}$, según corresponda).
\end{enumerate}
\fi

\def\parte{soluciones}
\ifx\capitulo\parte
\lstinputlisting[style=matlab-style]{snippets/ej13.m}
\fi
%%% Local Variables:
%%% mode: latex
%%% TeX-master: "../100Basicos"
%%% End:
