\subsection{Ejercicio 36}
\def\parte{enunciados}
\ifx\capitulo\parte
Implementa una función que devuelva el máximo de un vector, junto con la posición del vector en la que se encuentra. Emplea para ello estructuras repetitivas.
A continuación emplea esta función en un script para mostrar el máximo de un vector, y además cambiar todas las posiciones del vector, excepto la perteneciente al máximo, por cero.
\fi

\def\parte{recetas}
\ifx\capitulo\parte
Tenemos una función que recibe un solo parámetro, que es el vector al que le vamos a buscar el máximo, y que devuelve dos valores: el máximo del vector y su posición.

Lo que haremos en la función será recorrer los elementos del vector, e ir almacenando el máximo valor encontrado hasta el momento en la variable que devolverá el máximo, y también se devolverá la posición en la está ese elemento en la variable que devolverá la posición.

En el script que usa esta función simplemente:

\begin{enumerate}
\item Obtenemos el máximo y la posición del vector.
\item Mostramos el máximo.
\item Recorremos las posiciones del vector, y para aquellas que no sean la posición del máximo cambiamos el valor por cero. 
\end{enumerate}
\fi

\def\parte{soluciones}
\ifx\capitulo\parte
\lstinputlisting[style=matlab-style, caption={El archivo maxPos.m}]{snippets/maxPos.m}
\lstinputlisting[style=matlab-style, caption={El script que usa esta función.}]{snippets/ej36.m}
\fi
%%% Local Variables:
%%% mode: latex
%%% TeX-master: "../100Basicos"
%%% End:

%%% Local Variables:
%%% mode: latex
%%% TeX-master: "../100Basicos"
%%% End:
