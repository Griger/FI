\subsection{Ejercicio 3}
\def\parte{enunciados}
\ifx\capitulo\parte
La siguiente tabla muestra la nota de algunos estudiantes en diversas asignaturas sí:

\begin{center}
  \begin{tabular}[h]{cccc}
    &Matemáticas&Lengua&Filosofía \\
    Estudiante1&5&7&10 \\
    Estudiante2&6&4&3 \\
    Estudiante3&8&5&8 \\
    Estudiante4&9&8&9 
  \end{tabular}
\end{center}

\begin{itemize}
\item Calcula la desviación típica en la nota para cada una de las asignaturas.
\item Muestra la nota del Estudiante2 en Lengua.
\item Calcula la media de cada estudiante redondeándola al alza.
\item Muestra las notas de cada estudiante en un gráfico de barras.
\end{itemize}
\fi

\def\parte{recetas}
\ifx\capitulo\parte
\begin{enumerate}
\item Definimos la matriz.
\item Mostramos la nota del Estudiante2.
\item Calculamos la desviación típica en cada columna, es decir, a lo largo de la primera dimensión (a lo largo de las filas). Y mostramos el vector resultante.
\item Calculamos la media en cada fila, es decir, a lo largo de la segunda dimensión (se deja fija la fila y se mueve la columna). Y mostramos el vector resultante.
\item Mostramos la matriz empleando un gráfico de barras y añadimos título, etiquetas y leyenda para hacerlo más legible.
\end{enumerate}
\fi

\def\parte{soluciones}
\ifx\capitulo\parte
\lstinputlisting[style=matlab-style]{snippets/ej3.m}
\fi
%%% Local Variables:
%%% mode: latex
%%% TeX-master: "../100Basicos"
%%% End:
