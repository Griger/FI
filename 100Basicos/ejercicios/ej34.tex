\subsection{Ejercicio 34}
\def\parte{enunciados}
\ifx\capitulo\parte
Implementa una función que empleando estructuras repetitivas calcule la traza de una matriz no necesariamente cuadrada.
\fi

\def\parte{recetas}
\ifx\capitulo\parte
La función recibirá una matriz como argumento, y devolverá la traza de dicha matriz.

La traza de una matriz se puede ver como la suma de aquellos elementos cuya posición tiene la fila y columna igual. Entonces:
\begin{enumerate}
\item Calculamos el mínimo entre el número de filas y el de columnas de la matriz, así nos aseguraremos de no tener un fallo en las dimensiones ya que si por ejemplo tenemos una matriz con 2 filas y 3 columnas, el elemento M(3,3) no existe, y tendríamos un fallo en las dimensiones.
\item Teniendo en cuenta ese mínimo hacemos un bucle para sumar los elementos \code{M(i,i)} de la matriz.
\end{enumerate}
\fi

\def\parte{soluciones}
\ifx\capitulo\parte
\lstinputlisting[style=matlab-style]{snippets/ej34.m}
\fi
%%% Local Variables:
%%% mode: latex
%%% TeX-master: "../100Basicos"
%%% End:

%%% Local Variables:
%%% mode: latex
%%% TeX-master: "../100Basicos"
%%% End:
