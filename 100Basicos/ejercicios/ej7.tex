\subsection{Ejercicio 7}
\def\parte{enunciados}
\ifx\capitulo\parte
El número $\pi$ se puede aproximar empleando el conocido \href{https://en.wikipedia.org/wiki/Wallis_product}{Producto de Wallis} el cuál nos dice que:
\[\frac{\pi}{2} = \prod_{n = 1}^{\infty} \frac{4n^2}{4n^2 + 1}\]

Para comporobar que efectivamente sucesión converge a $\pi$:
\begin{itemize}
\item Calcula para cada natural $m$ desde 1 hasta 1000, el valor del productorio $2\prod_{n = 1}^{m} \frac{4n^2}{4n^2 + 1}$. Y observa cómo los valores se aproximan a $\pi$. Es importante que no metas el 2 dentro del productorio, es decir, primero se calcula el productorio y luego se multiplica por dos su valor. A diferencia de lo que ocurre con una sumatoria donde sí se podría meter dentro de la sumatoria algo que va multiplicándola.
\item Muestra una gráfica donde se muestren esos valores calculados y observa como los valores convergen a $\pi$.
\end{itemize}
\fi

\def\parte{recetas}
\ifx\capitulo\parte
\begin{enumerate}
\item Definimos un vector con los números del 1 al 1000.
\item Para cada valor $n$ en dicho vector calculamos el correspondiente término del productorio.
\item Ahora sobre ese nuevo vector con los términos calculamos otro vector con el producto acumulado. Así cada término se corresponderá con un productorio desde 1 hasta un cierto $n$.
\item Y multiplicamos ese vector por dos. Veremos cómo cada término se acerca más a $\pi$, es decir, cuántos más términos multiplicamos más nos acercamos a $\pi$.
\item Dibujamos los valores de este último vector en una gráfica.
\end{enumerate}
\fi

\def\parte{soluciones}
\ifx\capitulo\parte
\lstinputlisting[style=matlab-style]{snippets/ej7.m}
\fi
%%% Local Variables:
%%% mode: latex
%%% TeX-master: "../100Basicos"
%%% End:
