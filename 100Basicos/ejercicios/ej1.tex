\subsection{Ejercicio 1}
\def\parte{enunciados}
\ifx\capitulo\parte
Dado el siguiente vector con los radios de los círculos se pide:
\begin{itemize}
\item Mostrar el número de círculos que hay.
\item Obtener un vector con el área de cada círculo. Recordemos que la fórmula es $\pi r^2$, donde $r$ es el radio.
\item Obtener un vector con la longitud de la circunferencia de cada círculo. La fórmula para este cálculo es $2\pi r$.
\end{itemize}

El vector con los radios de ejemplo es el siguiente. Aunque se puede probar con otros vectores de distintas longitud o valores:

\begin{lstlisting}[style = matlab-style, numbers = none]
  v = [10 5 8 6 20 50];
\end{lstlisting}
\fi

\def\parte{recetas}
\ifx\capitulo\parte
\begin{enumerate}
\item Declaramos el vector con los radios.
\item Mostramos el número de círculos por pantalla, empleando la función que nos da el número de elementos en un vector.
\item Usamos la fórmula para calcular el área con el vector de radios, aprovechando cómo funcionan las operaciones con vectores en Matlab.
\item Hacemos lo propio para calcular la longitud de la circunferencia de cada círculo.
\end{enumerate}
\fi

\def\parte{soluciones}
\ifx\capitulo\parte
\lstinputlisting[style=matlab-style]{snippets/ej1.m}
\fi
%%% Local Variables:
%%% mode: latex
%%% TeX-master: "../100Basicos"
%%% End:
