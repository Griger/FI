\subsection{Ejercicio 6}
\def\parte{enunciados}
\ifx\capitulo\parte
Un estudiante ha obtenido las siguientes calificaciones en sus últimos exámenes: 5,6,8,4,3,8,7. Solicita al estudiante su nombre y muestra por pantalla:

\begin{enumerate}
\item La medida de sus notas.
\item Su mejor y peor nota.
\item Una gráfica mostrando la evolución de sus notas.
\end{enumerate}

Estos resultados se mostrarán siempre haciendo mención expresa al nombre del usuario. Por ejemplo, ``Fulanito, la media de tus notas es...''.
\fi

\def\parte{recetas}
\ifx\capitulo\parte
\begin{enumerate}
\item Definimos un vector que contenga las notas del estudiante. 
\item Le pedimos al usuario su nombre. Teniendo en cuenta que la información introducida por el usuario ha de tratarse de forma literal y no evaluarse (recuerda la opción de \code{input} que nos permite esto).
\item Mostramos por pantalla la media, el mínimo y el máximo de ese vector.
\item Dibujamos una gráfica con el valor de sus notas.
\end{enumerate}
\fi

\def\parte{soluciones}
\ifx\capitulo\parte
\lstinputlisting[style=matlab-style]{snippets/ej6.m}
\fi
%%% Local Variables:
%%% mode: latex
%%% TeX-master: "../100Basicos"
%%% End:
