\subsection{Ejercicio 31}
\def\parte{enunciados}
\ifx\capitulo\parte
Implementa una función \code{esPrimo} que, empleando estructuras repetitivas, devuelva un booleano indicando si un número es primo o no.

A continuación, implementa un script en el que, dado un número introducido por el usuario indique:
\begin{itemize}
\item Si es un número par y mayor que 10, o no.
\item En caso de serlo, se muestra si es un múltiplo de 5 o no.
\item Si no lo es, se imprime si es un número primo o no, empleando la función anteriormente implementada.
\end{itemize}
\fi

\def\parte{recetas}
\ifx\capitulo\parte
Para la función que comprueba si un número es primo. Esta función recibirá un número, y devolverá un booleano indicando si el número es primo o no. Entonces:

\begin{enumerate}
\item El booleano empezará siendo \code{true}.
\item Para cada entero mayor que 1 y menor que el número introducido, comprobamos si ese entero divide al número introducido, en cuyo caso el número tendría un divisor distinto del 1 y de él mismo, con lo que el booleano que indica si el número es primo o no pasaría a ser \code{false}.
\end{enumerate}

El script que se pide no es más que una estructura condicional anidada donde seguiremos las indicaciones que se dan en el enunciado para ver cómo anidamos las estructuras condicionales.
\fi

\def\parte{soluciones}
\ifx\capitulo\parte
\lstinputlisting[style=matlab-style, title=La función \texttt{esPrimo} que ha de guardarse en un archivo \texttt{esPrimo.m}.]{snippets/esPrimo.m}
\lstinputlisting[style=matlab-style, title=El script donde se usa la función \texttt{esPrimo}.]{snippets/ej31.m}
\fi
%%% Local Variables:
%%% mode: latex
%%% TeX-master: "../100Basicos"
%%% End:

%%% Local Variables:
%%% mode: latex
%%% TeX-master: "../100Basicos"
%%% End:
