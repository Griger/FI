\subsection{Ejercicio 15}
\def\parte{enunciados}
\ifx\capitulo\parte
Dada una matriz 3x3 (que la podemos definir como queramos)

\[
  \begin{pmatrix}
    a & b & c\\
    d & e & f\\
    g & h & i
  \end{pmatrix}
\]

Calcular su determinante haciendo uso de la siguiente fórmula:
\[det(A) = a(ei-fh)-b(di-fg)+c(dh-eg) \]
Comprueba si el resultado es correcto empleando la función \code{det}.
\fi

\def\parte{recetas}
\ifx\capitulo\parte
\begin{enumerate}
\item Definimos una matriz 3x3 como queramos.
\item Calculamos el determinando usando la formula anterior, en la que tendremos que acceder a los elementos de la matriz segun indica la formula.
\item Usamos la funcion \code{det} para calcular el determinante con la funcionalidad de \mt.
\end{enumerate}
\fi

\def\parte{soluciones}
\ifx\capitulo\parte
\lstinputlisting[style=matlab-style]{snippets/ej15.m}
\fi
%%% Local Variables:
%%% mode: latex
%%% TeX-master: "../100Basicos"
%%% End:
