\subsection{Ejercicio 27}
\def\parte{enunciados}
\ifx\capitulo\parte
Escribe una función que, dada una matriz devuelva:

\begin{itemize}
\item El número de elementos que son múltiplos de 2 o múltiplos de 3 (pero no de ambos), sin emplear estructuras repetitivas, es decir, empleando operadores relacionales.
\item El número de elementos que son múltiplos de 5 y de 4, empleando estructuras repetitivas.
\end{itemize}
\fi

\def\parte{recetas}
\ifx\capitulo\parte
\begin{enumerate}
\item La función a diseñar recibe como parámetro la matriz que vamos a analizar. Y en este caso es una función que devuelve \textbf{dos valores}.
\item Obtengo una matriz de booleanos donde se indique qué elementos de la matriz son múltiplos de 2 o de 3, pero no de ambos. Y uso esos booleanos (que son ceros y unos) para obtener el número total de elementos que cumplen esa condición, sumando los booleanos (esta suma será la que devuelva como uno de los dos valores devueltos por la función).
\item Para el segundo paso obtengo el número de filas y de columnas de la matriz en dos variables.
\item Con un bucle \code{for} doble recorro la matriz: el primer bucle recorrerá las filas, y el bucle interior las columnas. Para cada elemento de la matriz (\code{M(i,j)}), si cumple que es múltiplo de 5 y de 4, aumento en 1 una variable (que es la que devuelvo como segundo valor) que lleva la cuenta de los elementos de la matriz que cumplen esto.
\end{enumerate}
\fi

\def\parte{soluciones}
\ifx\capitulo\parte
\lstinputlisting[style=matlab-style]{snippets/ej27.m}
\fi
%%% Local Variables:
%%% mode: latex
%%% TeX-master: "../100Basicos"
%%% End:
