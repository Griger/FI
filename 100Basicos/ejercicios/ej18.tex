\subsection{Ejercicio 18}
\def\parte{enunciados}
\ifx\capitulo\parte
El siguiente vector recoge la altura de 10 personas:
\begin{center}
  \begin{tabular}[h]{cccccccccc}
    1.5&1.9&1.35&2&1.6&1.72&1.7&1.85&1.4&1.45
  \end{tabular}
\end{center}

Calcular:

\begin{itemize}
\item La altura de la persona más alta y de la más baja.
\item La media de los individuos que ocupan una posición múltiplo de tres.
\item La media de las 5 personas más bajas.
\item La suma de la altura de las 3 personas más altas.
\end{itemize}
\fi

\def\parte{recetas}
\ifx\capitulo\parte
\begin{enumerate}
\item Declarar el vector con las alturas.
\item Mostrar el máximo y el mínimo de ese vector.
\item Ordenar el vector de menor a mayor y calcular la media de las 5 primeras posiciones de ese vector ordenado.
\item Ordenar el vector de mayor a menor y calcular la suma de las 3 primeras posiciones.
\end{enumerate}
\fi

\def\parte{soluciones}
\ifx\capitulo\parte
\lstinputlisting[style=matlab-style]{snippets/ej18.m}
\fi
%%% Local Variables:
%%% mode: latex
%%% TeX-master: "../100Basicos"
%%% End:
