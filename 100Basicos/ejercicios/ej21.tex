\subsection{Ejercicio 21}
\def\parte{enunciados}
\ifx\capitulo\parte
Obten una matriz en la que:
\begin{itemize}
\item La primera fila contenga valores de 1 a 10 con saltos de 0.2.
\item La segunda contenga el valores de la función $\sqrt{x}$ sobre los valores de la primera fila.
\item La tercera contenga los valores de la función $sin(x)*cos(x)$ sobre los valores de la primera fila.
\item La cuarta fila contenga los valores de la función $log(x)$ sobre los valores de la primera fila.
\end{itemize}

Dibuja la gráfica de las 3 funciones empleando los datos contenidos en la matriz.
\fi

\def\parte{recetas}
\ifx\capitulo\parte
\begin{enumerate}
\item Definimos un vector con los valores de la primera fila.
\item Definimos la matriz haciendo uso de las operaciones vectoriales.
\item Dibujamos las gráficas dándole distintos colores a cada una, accediendo para ello a las filas correspondientes de la matriz.
\end{enumerate}
\fi

\def\parte{soluciones}
\ifx\capitulo\parte
\lstinputlisting[style=matlab-style]{snippets/ej21.m}
\fi
%%% Local Variables:
%%% mode: latex
%%% TeX-master: "../100Basicos"
%%% End:
