\subsection{Ejercicio 19}
\def\parte{enunciados}
\ifx\capitulo\parte
Genera un vector de 100 números aleatorios en el rango $[5,10]$ y dibuja los valores obtenidos. Para ello recuerda la siguiente formula cambia la escala, generando para cada número $x$ en un rango $[a,b]$, su correspondiente número en el rango $[c,d]$:
\[x \longrightarrow c + \frac{(d-c)(x - a)}{b-a}\]
\fi

\def\parte{recetas}
\ifx\capitulo\parte
\begin{enumerate}
\item Obtenemos un vector de 100 números aleatorios en el rango $[0,1]$ con la función.
\item Usamos sobre este vector la fórmula del enunciado donde ahora partirmos del rango $[0,1]$ y pasamos al rango $[5,10]$.
\item Pintamos los valores.
\end{enumerate}
\fi

\def\parte{soluciones}
\ifx\capitulo\parte
\lstinputlisting[style=matlab-style]{snippets/ej19.m}
\fi
%%% Local Variables:
%%% mode: latex
%%% TeX-master: "../100Basicos"
%%% End:
