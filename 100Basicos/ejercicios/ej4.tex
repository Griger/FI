\subsection{Ejercicio 4}
\def\parte{enunciados}
\ifx\capitulo\parte
La altura de una pelota de goma que se deja caer al suelo desde una cierta altura viene dada por la siguiente función, donde $t$ es el tiempo en segundos:
\[\mid sin(t^t) / 2^{(t^t - \frac{\pi}{2})/\pi} \mid\]
Pide al usuario un instante de tiempo entre el 1 y 3, y calcula la altura de la pelota en ese instante. A continuación visualiza gráficamente cómo evoluciona la altura de la pelota en ese rango de tiempo, y testea  mirando la gráfica si el valor obtenido se corresponde con la realidad.
\fi

\def\parte{recetas}
\ifx\capitulo\parte
\begin{enumerate}
\item Pedir al usuario el instante de tiempo $t$ en el que calcular la altura, y hacer los cálculos necesarios para obtener la altura de la pelota en dicho instante.
\item Mostrar el resultado por pantalla.
\item Construir un vector que contenga el rango de valores donde se va a evaluar la función para dibujarla (los valores del eje x). Emplear saltos suficientemente pequeños en dicho rango para que la función se dibujo con la suficiente ``resolución''.
\item Obtener el vector de valores $y$ a partir del vector anterior.
\item Dibujar la gráfica usando ambos vectores.
\end{enumerate}
\fi

\def\parte{soluciones}
\ifx\capitulo\parte
\lstinputlisting[style=matlab-style]{snippets/ej4.m}
\fi
%%% Local Variables:
%%% mode: latex
%%% TeX-master: "../100Basicos"
%%% End:
