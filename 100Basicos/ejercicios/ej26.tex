\subsection{Ejercicio 26}
\def\parte{enunciados}
\ifx\capitulo\parte
Dado un vector con los valores del 1 al 100, escribe un script en el que:

\begin{itemize}
\item Con estructuras repetitivas cambiar el signo de los múltiplos de 5.
\item Sin emplear estructuras repetitivas, empleando operadores relacionales, multiplicar por 2 los múltiplos de 3.
\end{itemize}
\fi

\def\parte{recetas}
\ifx\capitulo\parte
\begin{enumerate}
\item Definimos el vector.
\item Para la primera parte con un bucle \code{for} vamos recorriendo los elementos del vector, teniendo en cuenta que \textbf{necesitamos modificarlos}.
\item Para cada elemento del vector, si es múltiplo de 5, lo cambiamos de signo.
\item Para la segunda parte, accedemos a las posiciones donde hay valores múltiplos de 3, y le asignamos el resultado de multiplicar las mismas posiciones por 2.
\end{enumerate}
\fi

\def\parte{soluciones}
\ifx\capitulo\parte
\lstinputlisting[style=matlab-style]{snippets/ej26.m}
\fi
%%% Local Variables:
%%% mode: latex
%%% TeX-master: "../100Basicos"
%%% End:
