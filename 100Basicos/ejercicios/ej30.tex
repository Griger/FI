\subsection{Ejercicio 30}
\def\parte{enunciados}
\ifx\capitulo\parte
Implementa una función que calcule empleando estructuras repetitivas la media de los elementos de la triangular superior de una matriz cuadrada.
\fi

\def\parte{recetas}
\ifx\capitulo\parte
\begin{enumerate}
\item La función recibe como argumento la matriz a la que le vamos a calcular la media de su triangular superior. Al igual que en Matlab la función \code{mean} recibe por ejemplo el vector al que le vamos a calcular su media.
\item La función devolverá un valor; la media de la triangular superior de la matriz. Al igual que  la función \code{mean} devuelve la media del vector que le pasemos.
\item Para calcular la media recorremos los elementos de la triangular superior:
  \begin{itemize}
  \item En la fila i estos elementos serán los que van de la columna i hasta el final.
  \item Por tanto el primer bucle recorrerá las filas y el segundo, dentro del primero, recorrerá las columnas que vayan desde la columna i (si estamos en la fila i), hasta el final.
  \end{itemize}
\item Vamos acumulando la suma de estos elementos en una variable, y también llevamos la cuenta de cuántos elementos hemos sumado en otra variable. Con estos dos datos finalmente calculamos la media, que será el valor que devolveremos.
\end{enumerate}
\fi

\def\parte{soluciones}
\ifx\capitulo\parte
\lstinputlisting[style=matlab-style]{snippets/ej30.m}
\fi
%%% Local Variables:
%%% mode: latex
%%% TeX-master: "../100Basicos"
%%% End:
