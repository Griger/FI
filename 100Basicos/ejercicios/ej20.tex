\subsection{Ejercicio 20}
\def\parte{enunciados}
\ifx\capitulo\parte
Una empresa cuenta con varios empleados con los siguientes salarios:
\begin{center}
  \begin{tabular}[h]{cccccc}
    1100&2500&1350&2000&2000&1800
  \end{tabular}
\end{center}

Baraja los empleados aleatoriamente y:
\begin{itemize}
\item Aplica un aumento de un 10\% a los empleados que hayan quedado en posiciones pares.
\item Aumenta 100€ a los empleados en las 3 primeras posiciones.
\item Duplica el salario al empleado en la última posición.
\end{itemize}
\fi

\def\parte{recetas}
\ifx\capitulo\parte
\begin{enumerate}
\item Definir un vector con los salarios.
\item Obtener una permutación aleatoria de los números del 1 al 6.
\item Barajar el vector, usando la permutación anterior para acceder a los elementos del vector por ese orden.
\item Aplicar los cambios en el vector de salario una vez barajado.
\end{enumerate}
\fi

\def\parte{soluciones}
\ifx\capitulo\parte
\lstinputlisting[style=matlab-style]{snippets/ej20.m}
\fi
%%% Local Variables:
%%% mode: latex
%%% TeX-master: "../100Basicos"
%%% End:
