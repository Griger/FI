\subsection{Ejercicio 37}
\def\parte{enunciados}
\ifx\capitulo\parte
Implementa un script en el que se le pida al usuario su nombre, su edad, y su peso. Entonces:

\begin{itemize}
\item A los menores de edad se les solicitará su altura en metros, y se mostrará su IMC ($peso / altura^2$).
\item A los usuarios de entre 30 y 50 años se les solicitará su sueldo.
\item A aquellos que tienen menos de 18 años se les preguntará si desean estudiar un idioma o no. Deberán responder ``N'' o ``S'', y en base a esta respuesta se les sugerirá que estudien sueco o que elijan música como actividad extraescolar.
\item Dependiendo del sueldo del usuario, se les recomendará alquilar un piso, comprar un piso o comprar una casa:
  \begin{itemize}
  \item Menos de 700€: alquilar piso.
  \item Entre 700€ y 1500€: comprar piso.
  \item Más de 1500€: comprar casa.
  \end{itemize}
\item Para aquellas personas mayores de 50 años se les sugerirá un plan de pensiones. Además si su peso excede los 100kg, entonces se les recomendará hacer deporte.
\end{itemize} 
La información que se muestra por pantalla irá acompañada por el nombre del usuario.
\fi

\def\parte{recetas}
\ifx\capitulo\parte
En este ejercicio no tenemos más que seguir las indicaciones que se dan en el enunciado. Teniendo en cuenta:

\begin{itemize}
\item Que algunos puntos se pueden agrupar bajo la misma condición, y no hay que hacer dos bloques condicionales distintos.
\item Que queremos que la entrada del usuario a la pregunta de si quiere o no estudiar un idioma sea una cadena que puede ser solamente ``S'' o ``N'', con lo que habrá que forzar la entrada con un \code{while}.
\end{itemize}
\fi

\def\parte{soluciones}
\ifx\capitulo\parte
\lstinputlisting[style=matlab-style]{snippets/ej37.m}
\fi
%%% Local Variables:
%%% mode: latex
%%% TeX-master: "../100Basicos"
%%% End:

%%% Local Variables:
%%% mode: latex
%%% TeX-master: "../100Basicos"
%%% End:
