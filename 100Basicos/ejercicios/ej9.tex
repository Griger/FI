\subsection{Ejercicio 9}
\def\parte{enunciados}
\ifx\capitulo\parte
Un cuadrado mágico es una matriz cuadrada de un tamaño determinado, en la que aparecen todos los números de 1 hasta $n^2$ sin repetir, siendo $n$ el tamaño de la matriz. Además esta matriz cumple que la suma de los elementos en cada fila, la suma de los elementos de cada columna y la suma de los elementos de las dos diagonales principales de la matriz es la misma.

La función \code{magic} en Matlab nos genera un cuadrado mágico con el tamaño que nosotros le pasemos como argumento.

Genera una cuadrado mágico con dicha función y comprueba que sus filas, columnas y diagonales suman todas lo mismo.
\fi

\def\parte{recetas}
\ifx\capitulo\parte
\begin{enumerate}
\item Generamos la matriz con la función \code{magic}.
\item Calculamos la suma de cada una de sus filas.
\item Calculamos la suma de cada una de sus columnas.
\item Calculamos la suma de los elementos de sus diagonales principales: una será la traza de la matriz, y la otra la traza de la matriz resultante de voltear la matriz de izquierda a derecha.
\end{enumerate}
\fi

\def\parte{soluciones}
\ifx\capitulo\parte
\lstinputlisting[style=matlab-style]{snippets/ej9.m}
\fi
%%% Local Variables:
%%% mode: latex
%%% TeX-master: "../100Basicos"
%%% End:
