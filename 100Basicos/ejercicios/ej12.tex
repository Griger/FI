\subsection{Ejercicio 12}
\def\parte{enunciados}
\ifx\capitulo\parte
La siguiente igualdad es una generalización de la identidad de Euler, donde $n$ es un número natural mayor que 1, e $i$ es la unidad imaginaria:
\[\sum_{k = 0}^{n-1}e^{2\pi i \frac{k}{n}} = 0\]
Comprueba que esta identidad se da para $n = 100, 1000$ y $10000$.
\fi

\def\parte{recetas}
\ifx\capitulo\parte
\begin{enumerate}
\item Definir un vector con los números de 0 a 9999, que son los que vamos a necesitar en las sumatorias que vamos a calcular.
\item Calcular 3 vectores con los términos de cada una de las sumatorias. \textit{Observa que no se puede calcular un vector único de términos y luego sumar los 100 primeros o los 1000 primeros, puesto que para cada $n$ los términos de la sumatoria cambian ligeramente, ya que en el exponente tenemos $n$ dividiendo}.
  \begin{itemize}
  \item Tener en cuenta que la función \code{exp} aplicada sobre un vector actúa elemento a elemento, devolviendo un vector en el que cada elemento es el resultado de $e^x$ donde $x$ es cada uno de los elementos del vector.
  \end{itemize}
\item Mostrar para cada $n$ la suma de los términos que hemos calculado.
\end{enumerate}
\fi

\def\parte{soluciones}
\ifx\capitulo\parte
\lstinputlisting[style=matlab-style]{snippets/ej12.m}
\fi
%%% Local Variables:
%%% mode: latex
%%% TeX-master: "../100Basicos"
%%% End:
