\subsection{Ejercicio 25}
\def\parte{enunciados}
\ifx\capitulo\parte
La siguiente matriz muestra el precio base de distintos productos en distintos países:
\begin{center}
  \begin{tabular}[h]{cccc}
    &Producto A&Producto B&Producto C \\
    País 1&100&200&300 \\
    País 2&400&300&200 \\
    País 3&500&400&100
  \end{tabular}
\end{center}

Y la siguiente matriz recoge el factor a aplicar a cada uno de esos productos en cada país debido a distintas políticas internacionales:

\begin{center}
  \begin{tabular}[h]{cccc}
    &Producto A&Producto B&Producto C \\
    País 1&1.1&0.5&0.3 \\
    País 2&0.5&2&1.2 \\
    País 3&1&1&1.5
  \end{tabular}
\end{center}

Calcula:

\begin{itemize}
\item El precio medio del Producto A en los distintos países después de aplicar los factores.
\item La diferencia media absoluta en el precio del Producto B antes y después de aplicar los factores
\item El precio medio de los productos en el País 2 después de aplicar los factores.
\item La diferencia media absoluta de precios entre los países 1 y 3 después de aplicar los factores.
\end{itemize}
\fi

\def\parte{recetas}
\ifx\capitulo\parte
\begin{enumerate}
\item Definimos las matrices.
\item Calculamos el precio de cada producto en cada país después de aplicar los factores, calculando el producto elemento a elemento de ambas matrices.
\item Calculamos los 4 puntos que se piden en el ejercicio.
\end{enumerate}
\fi

\def\parte{soluciones}
\ifx\capitulo\parte
\lstinputlisting[style=matlab-style]{snippets/ej25.m}
\fi
%%% Local Variables:
%%% mode: latex
%%% TeX-master: "../100Basicos"
%%% End:
