\subsection{Ejercicio 29}
\def\parte{enunciados}
\ifx\capitulo\parte
Define dos funciones \code{maximum} y \code{minimum} que calculen, empleando estructuras repetitivas el máximo y el mínimo de un vector.

Emplea estas dos funciones en un script en el que se defina un vector a mano de números positivos. Luego se pedirá al usuario números mientras inserte números mayores o iguales que cero. Cuando el usuario deje de introducir números se mostrará la suma de aquellos números introducidos por el usuario que estan en el intervalo [mínimo, máximo]. Donde mínimo y máximo son el mínimo y el máximo del vector previamente definido.
\fi

\def\parte{recetas}
\ifx\capitulo\parte
Para definir una función que calcule con estructuras repetitivas el máximo de un vector:
\begin{enumerate}
\item La función recibe el vector como parámetro. Y devuelve un solo valor, que será el máximo del vector.
\item Entonces inicialmente el máximo será el primer elemento del vector.
\item Luego iremos recorriendo el vector. En este caso podemos recorrer directamente los valores sin hacer \code{v(i)} ya que sólo no nos interesan los valores del vector, y no vamos a modificar el vector en sí.
\item Para cada valor del vector, si es mayor que el máximo actual que hemos encontrado, entonces el máximo pasará a ser ese valor.
\end{enumerate}
La función para calcular el mínimo se definiría de forma análoga. Ahora para el script donde se usan las funciones:
\begin{enumerate}
\item Definimos un vector de números positivos, como queramos.
\item Le pedimos al usuario un primer valor.
\item Y mientras que el usuario introduzca valores mayor o iguales que cero:
  \begin{enumerate}
  \item Si el valor introducido está en el rango [minimo, maximo], lo sumamos a la suma actual de los valores en el rango.
  \end{enumerate}
\item Finalmente mostramos la suma calculada cuando el usuario haya introducido un valor negativo, parando así el bucle en el que se le van pidiendo valores.
\end{enumerate}
\fi

\def\parte{soluciones}
\ifx\capitulo\parte
\lstinputlisting[style=matlab-style, title=Archivo \texttt{maximum.m}]{snippets/maximum.m}
\lstinputlisting[style=matlab-style, title=Archivo \texttt{minimum.m}]{snippets/minimum.m}
\lstinputlisting[style=matlab-style, title=El script donde se usan ambas funciones.]{snippets/ej29.m}
\fi
%%% Local Variables:
%%% mode: latex
%%% TeX-master: "../100Basicos"
%%% End:
