\subsection{Ejercicio 40}
\def\parte{enunciados}
\ifx\capitulo\parte
Implemente una función que dados dos enteros positivos, devuelva el cociente y el resto de la división entera del primero por el segundo. Para ello se empleará el algoritmo de la división, que se describe como sigue:
\begin{enumerate}
\item El cociente comienza siendo 0, y el resto comienza siendo el dividendo de la división a realizar.
\item Mientras que el resto sea mayor o igual que el divisor.
  \begin{enumerate}
  \item El cociente aumenta en uno.
  \item Al resto se le resta el divisor.
  \end{enumerate}
\end{enumerate}
\fi

\def\parte{recetas}
\ifx\capitulo\parte
La función recibirá dos valores, el dividendo y el divisor. Y devolverá también dos valores, el cociente y el resto de la división entera. Para calcular ese cociente y el resto no tenemos más que implementar dentro de la función el algoritmo que se describe en el enunciado.

Recordemos que para recoger los valores de una función que devuelve varios valores entonces tenemos que hacer algo como \code{[Q,R] = ej40(10,5)}.
\fi

\def\parte{soluciones}
\ifx\capitulo\parte
\lstinputlisting[style=matlab-style]{snippets/ej40.m}
\fi
%%% Local Variables:
%%% mode: latex
%%% TeX-master: "../100Basicos"
%%% End:

%%% Local Variables:
%%% mode: latex
%%% TeX-master: "../100Basicos"
%%% End:
