\subsection{Ejercicio 8}
\def\parte{enunciados}
\ifx\capitulo\parte
Dado el siguiente vector
\begin{lstlisting}[style=matlab-style]
  v = [-1 5 6 8 9 10 20 -5 6 8];
\end{lstlisting}
Realiza las siguientes operaciones sobre este vector:
\begin{enumerate}
\item Muestra el vector en orden inverso.
\item Muestra el vector ordenado de menor a mayor.
\item Poner en valor absoluto los elementos en las posiciones impares.
\item Duplicar los elementos en las posiciones múltiplo de 3.
\item Obtener la suma de los 5 primeros elementos.
\item Obtener el producto de los elementos en posiciones pares.
\item Restar 1 a los 5 últimos elementos del vector.
\end{enumerate}
\fi

\def\parte{recetas}
\ifx\capitulo\parte
Para este ejercicio sólo conviene apuntar el modo más adecuado de obtener los $m$ últimos elementos de un vector (siempre que el vector tenga al menos $m$ elementos). Como sabemos \code{end} nos devuelve la última posición del vector, es decir, si un vector tiene 10 elementos, 10 posiciones, \code{end} valdrá 10. Por otro lado si el vector tiene 15 elementos \code{end} vale 15. Entonces la posición \code{end - h} sería la posición \texttt{h} posiciones antes que la última posición. Por tanto los últimos 5 elementos del vector se accederían con \code{v(end-4:end)}.
\fi

\def\parte{soluciones}
\ifx\capitulo\parte
\lstinputlisting[style=matlab-style]{snippets/ej8.m}
\fi
%%% Local Variables:
%%% mode: latex
%%% TeX-master: "../100Basicos"
%%% End:
