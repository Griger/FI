\subsection{Ejercicio 5}
\def\parte{enunciados}
\ifx\capitulo\parte
Dadas dos matrices 5x12 con elementos aleatorios en el intervalo [0,1000]. Calcular, suponiendo que la primera matriz representa los gastos de 5 personas en cada uno de los meses del año, y la segunda representa sus ingresos:
\begin{enumerate}
\item Para cada persona y cada mes la cantidad (negativa o positiva) neta que ha entrado en su cuenta en ese mes ($ingresos - gastos$).
\item El balance anual para cada persona (la suma de los ingresos en cada mes para cada persona).
\item El neto máximo mensual de cada persona. Y el neto mínimo.
\item La media de gastos anual entre todas las personas.
\item El ratio ingresos/gastos mensual de cada persona.
\end{enumerate}
\fi

\def\parte{recetas}
\ifx\capitulo\parte
\begin{enumerate}
\item Generar las matrices aleatorias, que tendrán valores en el rango $[0,1]$. Ahora para tener valores en el rango $[0,1000]$, no tenemos más que multiplicar esos elementos por $1000$.
\item Calcular la diferencia de las dos matrices para obtener una matriz con el neto de cada mes para cada persona.
\item Hacer la suma por filas de la matriz anterior para obtener el neto anual, es decir, a lo largo de la segunda dimensión.
\item Sobre la matriz del segundo punto obtener el máximo y el mínimo de la fila, es decir, a lo largo de la segunda dimensión.
\item Obtener el gasto anual de cada persona y hacer la media de los valores obtenidos.
\item Hacer la división de la matriz de ingresos por la matriz de gastos, elemento a elemento.
\end{enumerate}
\fi

\def\parte{soluciones}
\ifx\capitulo\parte
\lstinputlisting[style=matlab-style]{snippets/ej5.m}
\fi
%%% Local Variables:
%%% mode: latex
%%% TeX-master: "../100Basicos"
%%% End:
