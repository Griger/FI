\subsection{Ejercicio 23}
\def\parte{enunciados}
\ifx\capitulo\parte
Dadas las dos matrices siguientes:
\[
  A = \begin{pmatrix}
    2&3&-5\\
    10&1&0\\
    5&5&6
  \end{pmatrix}
  B = \begin{pmatrix}
    4&0&1\\
    -9&-2&5\\
    0&1&7
  \end{pmatrix}
\]

Calcula:
\begin{itemize}
\item El producto de $A$ por $B$, y de $B$ por $A$.
\item La matriz resultante de multiplicar las matrices elemento a elemento.
\item $A*A*B$, o lo que es lo mismo $A^2B$
\item El producto de $A$ por su matriz inversa.
\item La suma de las dos matrices.
\item La matriz resultante de multiplicar los elementos de $B$ por el determinante de $A$.
\item La matriz resultante de dividir los elementos de $A$ por la traza de $B$.
\end{itemize}
\fi

\def\parte{recetas}
\ifx\capitulo\parte
Este ejercicio no precisa receta, salvo destacar la diferencia entre las instrucciones \code{A.^2} y \code{A^2}, donde la primera nos da el la matriz resultante de elevar cada elemento de \code{A} al cuadrado, y la segunda el resultado de multiplicar \code{A} por ella misma. Siendo la última la que hemos de emplear en este ejercicio.
\fi

\def\parte{soluciones}
\ifx\capitulo\parte
\lstinputlisting[style=matlab-style]{snippets/ej23.m}
\fi
%%% Local Variables:
%%% mode: latex
%%% TeX-master: "../100Basicos"
%%% End:
