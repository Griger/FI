\subsection{Ejercicio 24}
\def\parte{enunciados}
\ifx\capitulo\parte
En \mt existen dos funciones que nos generan números aleatorios en el intervalo $[0,1]$. \code{rand} genera valores en el este intervalo siguiendo una distribución uniforme, es decir, todos los valores tienen la misma probabilidad de salir. Por otro lado \code{randn} genera valores siguiendo una distribución normal de media 0 y varianza 1, donde los valores más probables son aquellos más cercanos a la media, más cercanos a cero (puesto que esta distribución tiene la forma de lo que se conoce como una campana de Gauss centrada en el cero):

\begin{itemize}
\item Obten un vector de 1000 números aleatorios con cada una de estas funciones.
\item Calcula la media de cada uno de los vectores.
\end{itemize}

Verás como la media del vector generado con \code{rand} está cercana a $0.5$; se obtienen valores distribuidos uniformemente a lo largo de todo el intervalo $[0,1]$, cuyo punto medio es el $0.5$. Por otro lado la media del vector obtenido con \code{randn} estará cercana a $0$, que como hemos dicho anteriormente es la media que tiene la distrubución que siguen los valores generados con esta función. Puedes repetir el experimento tantas veces como desees.
\fi

\def\parte{recetas}
\ifx\capitulo\parte
Este ejercicio no precisa receta.
\fi

\def\parte{soluciones}
\ifx\capitulo\parte
\lstinputlisting[style=matlab-style]{snippets/ej24.m}
\fi
%%% Local Variables:
%%% mode: latex
%%% TeX-master: "../100Basicos"
%%% End:
