\subsection{Ejercicio 11}
\def\parte{enunciados}
\ifx\capitulo\parte
Solicitar al usuario su nombre y su saldo actual. Supuesto que un lápiz cuesta 1.5€ y una goma de borrar cuesta 2€. Calcula:
\begin{itemize}
\item Cuál es el número máximo de lápices que puede comprar, y cuántas gomas puede comprar con el dinero sobrante.
\item Y viceversa: cuál es el número máximo de gomas que puede comprar, y cuántos lápices puede comprar con el dinero sobrante.
\end{itemize}
Dirígete al usuario empleando su nombre.
\fi

\def\parte{recetas}
\ifx\capitulo\parte
\begin{enumerate}
\item Obtener el resto y el cociente de la división entera de dividir el saldo por el precio de un lápiz: el cociente será el número máximo de lápices y el resto el dinero sobrante. Con este dinero sobrante se calculan cuántas gomas se pueden comprar, nuevamente calculando el cociente de la división entera.
\item El segundo apartado se hace de forma análoga.
\end{enumerate}
\fi

\def\parte{soluciones}
\ifx\capitulo\parte
\lstinputlisting[style=matlab-style]{snippets/ej11.m}
\fi
%%% Local Variables:
%%% mode: latex
%%% TeX-master: "../100Basicos"
%%% End:
